\documentclass{standalone}

\begin{document}

\section[Cytokinoma Dataset]{Cytokinoma dataset}\label{cytokine}

Increasing evidence suggests that inflammation is involved in Alzheimer's disease (AD) pathogenesis.
Elevated peripheral levels of different cytokines and chemokines in subjects affected by AD compared with healthy control (HC) have emphasized the role of peripheral inflammation in the disease.
Thus, these proteins can represent specific factors of disease development and progression.
Considering the cross-talking between the central nervous system and the periphery, the inflammatory analytes may provide utility as biomarkers to identify AD at earlier stages, in particular for the diagnosis of Mild Cognitive Impairment (MCI), a condition at risk of development of dementia.
An emerging field in the study of neuroinflammation is the sex-related differences: in the last years, gender studies have been increasingly developed with the aim to adopt gender differences as a key to interpretation many diseases, including neurodegenerative diseases.

The interactive network of cytokines/chemokines, defined as \quotes{cytokinome}, is extremely complex.
Using the DNetPRO algorithm as statistical feature selection method, we might discriminate the groups and propose a useful tool to follow the progression and evolution of AD from its early stages, also in light of gender differences.

\end{document}
