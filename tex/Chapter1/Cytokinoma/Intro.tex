\documentclass{standalone}

\begin{document}

\section[Cytokinoma Dataset]{Cytokinoma dataset}\label{cytokine}

Increasing evidence suggests that inflammation is involved in Alzheimer's disease (AD) pathogenesis.
Elevated peripheral levels of different cytokines and chemokines in subjects affected by AD compared with healthy control (CTL) have emphasized the role of peripheral inflammation in the disease.
Thus, these proteins can represent specific factors of disease development and progression.
Considering the cross-talking between the central nervous system and the periphery, the inflammatory analytes may provide utility as biomarkers to identify AD at earlier stages, in particular for the diagnosis of Mild Cognitive Impairment (MCI), a condition at risk of development of dementia.
AD is a major neurocognitive disorder and the most common cause of dementia in the old age, accounting for 60\% to 80\% of all causes.
During the past decade, a conceptual shift occurred in the field of AD considering the disease as a continuum.
In this context, there is an urgent need for biomarkers identification able to accurately detect AD in an early stage, before the appearance of neurologic signs.
An early diagnosis can hopefully lead to a better and more effective treatment, which could potentially limit neuronal damage and prevent the development of overt AD.
An emerging field in the study of neuroinflammation is the sex-related differences: in the last years, gender studies have been increasingly developed with the aim to adopt gender differences as a key to interpretation many diseases, including neurodegenerative diseases.

Experimental data showed that many mechanisms are involved in AD pathogenesis including neuroinflammation.
The dysregulation of cytokines and chemokines is a central feature in the development of neuroinflammation, neurodegeneration, and demyelination both in the central and peripheral nervous systems.
Among many chemokines and cytokines, pro-inflammatory IFN$\alpha$2, TNF$\alpha$, and IL-1$\alpha$ are described as heterogeneously implicated in AD pathogenesis.

The interactive network of cytokines/chemokines, defined as \quotes{cytokinome}, is extremely complex.
Using the DNetPRO algorithm as statistical feature selection method, we might discriminate the groups and propose a useful tool to follow the progression and evolution of AD from its early stages, also in light of gender differences.
With this study, we aimed first at the identification of a potential proteins profile able to discriminate AD, MCI and CTL and, therefore identify a potential early and easy to get a diagnostic marker of subjects at risk.

\end{document}
