\documentclass{standalone}

\begin{document}

\section[Bovine Dataset]{Bovine Dataset}\label{bovine:bovine}

Paratuberculosis or Johne's disease (JD) in cattle is a chronic granulomatous gastroenteritis caused by infection with \emph{Mycobacterium avium subspecies paratuberculosis} (MAP).
JD is not treatable; therefore the early identification and isolation of infected animals is a key point to reduce its incidence worldwide.
In this work \textsf{DNetPRO} algorithm was applied to RNAseq experimental data of 5 cattle positive to MAP infection compared to 5 negative uninfected controls.
The purpose was to find a small set of differentially expressed genes able to discriminate between infected animals in a pre-clinical phase.
Results of the \textsf{DNetPRO} algorithm identified a small set of 10 transcripts that differentiate between potentially infected, but clinically healthy, animals belonging to paratuberculosis positive herds and negative unexposed animals.
Furthermore, the same set of 10 transcripts differentiate negative unexposed animals from positive animals based on the results of the ELISA test\footnote{
  The enzyme-linked immunosorbent assay.
  It is a common diagnostic tool as well as a quality control check in various bio-medical industries and in medicine.
} for bovine paratuberculosis and fecal culture.
Within the 10 transcripts that together had good discriminative potential, 5 (TRPV4, RIC8B, IL5RA, ERF and CDC40) show significant differential expression between the three groups while the remaining 5 transcripts (RDM1, EPHX1, STAU1, TLE1, ASB8) did not show a significant differences in at least one of the pairwise comparisons.
In conclusion, the discriminant analysis described here identified a set of 10 genes that discriminate between the exposed and sero-converted animals.
When tested in a larger cohort, these finding lead the possible use of RNA expression analysis as new diagnostic test for paratuberculosis.
Such a signature could allow early interventions to reduce the sanitary and economic burden, and to reduce the risk of infection spreading.

In the next sections a description of the dataset and of main \textsf{DNetPRO} results will be discussed.
Further information can be found in the original paper~\cite{Malvisi2019}.

\end{document}
