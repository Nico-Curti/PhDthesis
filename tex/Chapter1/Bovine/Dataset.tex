\documentclass{standalone}

\begin{document}

\subsection[Dataset]{Dataset}\label{bovine_data}

Paratuberculosis or Johne's disease (JD) in cattle is a chronic granulomatous gastroenteritis caused by infection with \emph{Mycobacterium avium subspecies paratuberculosis} (MAP).
JD is present worldwide, is a welfare issue and causes significant economic losses.
Cattle are usually infected as young calves but typically do not show clinical signs before 24 months of age, however not all infected animals progress to clinical disease.
JD is not treatable, therefore the early identification and isolation of infected animals, before they start shedding the bacteria, is a key point to reduce its incidence in cattle herds worldwide.
In addition, an association between MAP and Crohn's disease (CD) in humans has been suggested and intensively explored.
Given the economic losses and welfare concerns for livestock, and possible human health risk, the research interest in JD has been driven by the substantial difficulty in early diagnosis of infected animals and the exploration of potentially new diagnostic techniques.

The dataset used in this work was previously discussed and generated by some of the authors of the original paper.
In detail, the dataset used comprised 15036 transcripts from 15 samples, classified as \quotes{serologically negative non exposed cows/healthy} (5 samples, labeled as NN), \quotes{serologically negative exposed cows/ infected} (5 samples, NP) and \quotes{serologically positive cows/clinical} (5 samples, PP).
Only transcripts with non-zero measures for all samples were considered, reducing the dataset to 13529 transcripts.

All data generated or analyzed during this study is available upon request, furthermore all transcript counts per sample are given as supplementary information files of the original paper.


\end{document}
