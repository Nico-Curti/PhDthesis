\documentclass{standalone}

\begin{document}

\subsection[Signature Overlap]{Characterization of signature overlap}\label{synapse:overlap}

In the analysis of the Synapse dataset we used a complex pipeline of cross-validation (ref Fig.~\ref{fig:dnet_pipe}) to obtain a sufficient statistics.
The DNetPRO algorithm was designed to work on a single dataset since the signature extraction can involve different variables for different data subdivisions.
In our application we divide the dataset into a training-test subdivision and the signature were extracted along a 10-fold cross-validation over the training set.
This kind of setup could produce 10 totally different signatures, in the worst case.
Moreover we replicated our simulation for 100 repetitions and thus a set of 1000 totally independent signatures were extracted.

Starting from this large subset of variables we can evaluate the robustness of the DNetPRO algorithm in the variable identifications and thus we can evaluate the overlap between these signatures.
From a statistical point-of-view is quite unlikely that the same set of variables were included into all the extracted signatures, especially on this application in which the variable roles are assumed by genes.
On the other hand the overlap of these signatures could highlight a statistical significance of some variables and thus genes related to the understudied tumors.

As case study we analyzed only the KIRC mRNA dataset in which the extracted signatures ranged from 4 to 650 genes ($\mu=382$ genes).
For each gene we counted its occurrences along the 1000 signatures.
The same analysis was performed taking into account the signatures generated using the $K$-best score variables (ref.~\ref{dnetpro:toy} for further informations) and a random features extraction.
In Fig.~\ref{fig:overlap} the genes distribution obtained by the three methods are shown.

\begin{figure}[htbp]
\centering
\def\svgwidth{0.4\textwidth}
\input{./img/DNetPRO_overlap.pdf_tex}
\caption{
}
\label{fig:overlap}
\end{figure}



% DNetPRO unique genes 5694
% KBest unique   genes 2485 ---> half than DNetPRO

% Minimum Size Signature: 4
% Maximum Size Signature: 654
% Average Size Signature: 382.344
% Median  Size Signature: 371.500


% The signature overlap could be estimated either across the 10 data subdivisions related to the cross-validation step, or across the 100 simulations done.
% %The first case allows us to study the robustness of our \emph{best} signatures in relation of the data subdivision, while the second one cover the possibility of having the same data in the training section.
% Moreover, we also considered the overlap between all 1000 signatures: a part of the results obtained on the mRNA datasets are shown in Fig.~\ref{supfig:overlap}.
% In the first column (Fig.~\ref{supfig:overlap} (a-c-e-g)) the full combination of signatures was computed for each tumor type and the overlap was estimated as the number of genes that appear in both signatures, normalized by the size of the smallest one.
% In the second column (Fig.~\ref{supfig:overlap} (b-d-f-h)) the overlap of signatures across the 10-fold cross-validation was estimated, and only the genes belonging to the 10 signature intersection was considered.
% Also in this case we normalized the overlap by the smallest size of the signatures.


% % KIRC:total genes (100x10): 5694   mean # in fold (100): 1333.26
% % GBM: total genes (100x10): 8655   mean # in fold (100): 2113.73
% % LUSC:total genes (100x10): 7795   mean # in fold (100): 1304.92
% % OV : total genes (100x10): 8431   mean # in fold (100): 2448.49

% The highest result of overlap for all 1000 signatures was obtained in the KIRC dataset, with an average overlap of 57\%.
% Significant overlaps were obtained also in the GBM and OV datasets, in which the overlap is around 50\%.
% The worst percentage of overlap between signatures was obtained in the GBM dataset which is however around the 40\% on average.
% Moreover, the maximum overlap reached 100\%, except for the OV dataset in which it was 76\%.%\textbf{e il massimo quale é?}.%%% DONE
% %This behavior could be associated to the lower performance obtained on this dataset, possibly due to a high noise level in the data.

% %The GBM tumor is a well known difficult pathology and a low statistical agreement between the features extracted was expected.

% Analogous results about the tumor types was obtained on the analysis of the percentage of gene overlap across the cross-validation subdivision.
% We remark that we considered only the genes overlapping in the whole set of signatures extracted.
% Also in this case the most prominent results were obtained on the KIRC signatures with an average of 20\% of overlap in a 30\% of the signatures, with a peak of 34\%.
% GBM and OV show a less overlap percentage (around the 10\%) in the same (30\%) of the signatures, with a peak of 16\% and 27\% respectively.
% The worst results were obtained on the LUSC dataset where only in two cases the 11\% of overlap was reached.

% The signature intersection across the 100 simulations was analyzed with the same hard threshold: we considered only the genes belonging to the full set of 100 signatures in the same fold.
% The overlaps in this case does not give us percentages greater than 4\%.
% Reducing this condition to the 75\% of occurrences the overlap percentages reaches 42\% in KIRC and about 25\% in the other datasets.



\end{document}
