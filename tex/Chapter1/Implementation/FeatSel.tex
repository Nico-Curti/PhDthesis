\documentclass{standalone}


\begin{document}

\subsection[Network Signature]{Network signature}\label{implementation:network}

After the rearrangement of feature pairs in ascending order we can start to create the variable network and looking for its connected components as putative signatures.
Each feature will be represent a node in the network and a given pair will be a connection between them.
Since the full storage of the network would require a matrix $(N\times N)$ we have to chose a better strategy for the processing\footnote{
  We are working in the hypothesis of very large $N$.
}.

The ordered set of couples computes in the previous section represents a so-called \emph{COO sparse matrix} (Coordinate Format sparse matrix) and we can reasonable assume that the desired signature will be composed by the top ranking of them.
So, the first step will be to cut a reasonable percentage of the full set of pairs and process only them.

Moreover we are interested in a small set of variables unknown a prior.
The load of all the node pairs into the same graph can slow down the computation.
An iterative method (with stop criteria) can perform better in the large case of samples and only in worst cases the full set of pair will be loaded.

Since the described algorithm step does not require particular performance efficiency now, the main code used in our simulation was written in pure Python.
A C++ implementation of the same algorithm was developed with the help of the Boost Graph Library~\cite{BGL} (BGL) but to not overweight the code installation was reserved just for a style exercise.
In this section we discuss about this second version and about the strategies chosen to implement an efficiency version of it.
This version of the algorithm was also used as stand alone method for other applications that will be presented later.

BGL is a very wide framework for graph analysis based on template structures.
The library efficiency discourage the users to re-implement the same algorithms and for the current purposes it was resulted more than sufficient.

Starting from the top scorer feature pairs we progressively add each couple of nodes to an empty graph.
At each iteration step the number of connected components is evaluated until a desired number of nodes (greater or equal) is not reached\footnote{
  This procedure is quite similar to put a threshold value on the couple performances or just simpler highlight inside the full network the components linked by weights greater than a given value.
}.
Two degree of freedom are left to the user: in order, \textsf{pruning} and \textsf{merging}.
The first one perform an iteratively remotion of nodes with degree equal (or lower) than 1, i.e pendant nodes, until the graph core is not filtered.
The \textsf{merging} clause choose between the biggest connected component or the the set of all the disjoint connect components as putative signature.
The output of \textsf{merging} step determine the number of nodes in the graph which have been considered for the stop criteria.

A crucial role in the optimization of the algorithm is played by the BGL graph structure chosen.
Since the two degree of freedom imply a continuous rearrangement of the graph nodes the strategy chosen is to apply a filter mask over the main graph structure that highlights the only part of interest.
This can be done using the \textsf{boost :: filtered\_graph} object of BGL.
In ~\ref{code:featuresel} the C++ snippet is shown.

\lstset{style=c++}
\begin{lstlisting}[language=C++, caption=DNetPRO signature extraction, label=code:featuresel]
#include <boost/graph/adjacency_list.hpp>
#include <boost/graph/connected_components.hpp>
#include <boost/graph/filtered_graph.hpp>
#include <boost/function.hpp>
#include <boost/graph/iteration_macros.hpp>

typedef typename boost :: adjacency_list< boost :: vecS, boost :: vecS, boost :: undirectedS, boost :: property< boost :: vertex_color_t, int >, boost :: property < boost :: edge_index_t, int > > Graph;
using V = Graph :: vertex_descriptor;
using Filtered = boost :: filtered_graph < Graph, boost :: keep_all, boost :: function < bool(V) > >;


std :: vector < int > FeatureSelection (int ** couples, const int & min_size, bool pruning=true,  bool merging=true)
{
  Graph G;
  std :: vector < V > removed_set;
  Filtered Signature (G, boost :: keep_all {}, [] (V v) {return removed_set.end() == removed_set.find(v);});

  int L = 0, leave, Ncomp, i = 0;

  while ( true ){

    boost :: add_edge (couples[i][0], couples[i][1], G);

    while ( pruning ){

      leave = 0;
      BGL_FORALL_VERTICES (v, Signature, Filtered);
        if ( boost :: in_degree (v, Signature) < 2 ){
          removed_set.insert (v);
          ++ leave;
        }

      if ( leave == 0 )
        break;
    }

    if ( num_vertices (G) - removed_set.size() ){

      components.resize (num_vertices (G));

      Ncomp = boost :: connected_components (Signature, &components[0]);

      if ( merging ){

        BGL_FORALL_VERTICES (v, Signature, Filtered)
          core.push_back ( static_cast < int >(v) );
      }
      else {

        std :: map < int, int > size;
        for ( auto && comp : components ) ++ size[comp];

        auto max_key = std :: max_element (std :: begin(size), std :: end(size), [] (auto && p1, auto && p2) { return p1.second < p2.second; })->first;

        BGL_FORALL_VERTICES (v, Signature, Filtered)
          if ( components[v] == max_key )
            core.push_back( static_cast < int >(v) );
      }

      components.resize (0);
      L = static_cast < int >(core.size());
    }

    removed_set.clear();

    if ( L >= min_size ) break;

    ++ i;

    core.resize (0);
  }

  return core;
}

\end{lstlisting}

From the above description should be clear that given any set of ordered (in ascending order) couples of variables, this algorithm allows to extract the core network independently by the procedure which generate them.
So it can be used as dimensionality reduction algorithm of general purpose network structures.
An example of this kind of application was reported in Appendix B - Venice Road Network in which we summarized the results of~\cite{Mizzi2018, CurtiSDPS2018}.

\end{document}

