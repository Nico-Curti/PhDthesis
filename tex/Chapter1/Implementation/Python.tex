\documentclass{standalone}


\begin{document}


\subsection[Python wrap]{DNetPRO in Python}\label{implementation:python}

Up to now we have focused on the algorithm performances, leaving out the usability of the \textsf{DNetPRO} algorithm for the (research) community.
Despite the \textsf{C++} is one of the most efficient and older programming language\footnote{
  Still in common use in scientific research groups.
}, the number of \textsf{Python}-users is growing in the last years.
\textsf{Python} is becoming leader in scientific research publications and the major part of Machine Learning analyses are performed using \textsf{Python} libraries (in particular \textsf{scikit-learn} library).
So, we have to reach a compromise between performances and usability of the new codes and a reasonable solution is given by the \textsf{Cython}~\cite{behnel2010cython} language.

\textsf{Cython} \quotes{language}\footnote{
  It is not a real programming language since it is based on \textsf{Python}.
  However, it has its own syntax and keywords which are different either from \textsf{Python} and \textsf{C++}.
  It also needs a compiler to run and it is certainly different from \textsf{Python}.
} allows an easy interface between \textsf{C++} codes and \textsf{Python} language.
With a relatively simple wrapping of the \textsf{C++} functions, they can be used inside a pure \textsf{Python} code, preserving as much as possible the computational performances of a pure \textsf{C++} version.
In this way, we have written a simple \textsf{Python} object which performs the full set of \textsf{DNetPRO} steps and, moreover, which is compatible with the functions provided by other machine learning libraries.

With these purposes we have chosen to operate a \quotes{double wrap} of the \textsf{C++} functions to separate as much as possible the \textsf{C++} components from \textsf{Python}\footnote{
  \textsf{Cython} wraps are very powerful tools for \textsf{C++} integration into \textsf{Python} code but, by experience, they are difficult to manage by pure-\textsf{Python}-users.
  A simple workaround is to perform a first wrap of the \textsf{C++} functions inside a \textsf{Cython} object, adding a second wrap of it into a pure-\textsf{Python} class.
  This two-steps wrap certainly gets worse the computational performances, but it allows a complete separation between the compiled part of the code (\textsf{Cython}) and the interpreted (\textsf{Python}) one.
  Moreover, we can leave back all the checks on input parameters of the \textsf{C++} function since they can be performed at run time by the \textsf{Python} wrap.
}.
The \href{https://github.com/Nico-Curti/DNetPRO/blob/master/DNetPRO/DNetPRO.py}{\textsf{Python} object} was written considering a full compatibility with the \textsf{scikit-learn} library to allow the usage of the \textsf{DNetPRO} feature selection method as an alternative component of other Machine Learning pipelines.


\end{document}
