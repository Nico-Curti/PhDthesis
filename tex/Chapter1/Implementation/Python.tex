\documentclass{standalone}


\begin{document}


\subsection[Python wrap]{DNetPRO in Python}\label{implementation:python}

Up to now we are focusing on the algorithm performances, leaving out the usability of the \textsf{DNetPRO} algorithm for the (research) community.
Despite the \textsf{C++} is one of the most efficient and old programming language\footnote{
  Still in common use in scientific research groups.
}, the \textsf{Python} language users are increasing in the last few years.
\textsf{Python} is becoming a leader in scientific research publications and the large part of Machine Learning analysis are performed using \textsf{Python} libraries (in particular \textsf{scikit-learn} library).
So we have to reach a compromise between the performances and usability of new developed codes and it can be reached using the \textsf{Cython}~\cite{behnel2010cython} framework.

\textsf{Cython} \quotes{language}\footnote{
  It is not a real programming language since it is based on \textsf{Python}.
  However it has its own syntax and keywords which are different either from \textsf{Python} either from \textsf{C++}.
  In the end it needs a compiler to run and it is certainly different from \textsf{Python}.
} allows an easy interface between \textsf{C++} codes and \textsf{Python} language.
With a relatively simple wrapping of the \textsf{C++} functions, they can be used inside a pure \textsf{Python} code preserving as much as possible the computational performances of the pure \textsf{C++} version.
In this way we can create a simple \textsf{Python} object which performs the full set of \textsf{DNetPRO} steps and, moreover, which is compatible with the functions provided by the other machine learning libraries.

With this purposes we chose to operate a double wrap of the \textsf{C++} functions to separate as much as possible the \textsf{C++} component from the \textsf{Python} one\footnote{
  \textsf{Cython} wrap are very powerful tools for \textsf{C++} integration into \textsf{Python} code but, by experience, they are difficult to manage by pure-\textsf{Python} users.
  A simple workaround is to perform a first wrap of the \textsf{C++} function inside a \textsf{Cython} object and a second wrap of it into a pure-\textsf{Python} one.
  This two-steps wrap certainly gets worse the computational performances but it allows a complete separation between the compiled part of the code (\textsf{Cython}) and the interpreted (\textsf{Python}) one.
  Moreover we can leave back all the checks on input parameters
  in the \textsf{C++} version since they will be performed at run time in the \textsf{Python} wrap.
}.
The \href{https://github.com/Nico-Curti/DNetPRO/blob/master/DNetPRO/DNetPRO.py}{\textsf{Python} object} was written considering a full compatibility with the \textsf{scikit-learn} library to allow the use of the \textsf{DNetPRO} feature selection as an alternative component of other Machine Learning pipelines.


\end{document}
