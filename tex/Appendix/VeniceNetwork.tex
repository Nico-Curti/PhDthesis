\documentclass{standalone}

\begin{document}

\chapter*{Appendix B - Venice Road Network}

Tourist flows in historical cities are continuously growing in a globalized world and adequate governance processes, politics and tools are necessary in order to reduce impacts on the urban livability and to guarantee the preservation of cultural heritage.
The ICTs offer the possibility of collecting large amount of data that can point out and quantify some statistical and dynamic properties of human mobility emerging from the individual behavior and referring to a whole road network.
In this work we analyze a new dataset that has been collected by the Italian mobile phone company TIM, which contains the GPS positions of a relevant sample of mobile devices when they actively connected to the cell phone network.
Our aim is to propose innovative tools allowing to study properties of pedestrian mobility on the whole road network.
Venice is a paradigmatic example for the impact of tourist flows on the resident life quality and on the preservation of cultural heritage.
The GPS data provide anonymized geo-referenced information on the displacements of the devices.
After a filtering procedure, we develop specific algorithms able to reconstruct the daily mobility paths on the whole Venice road network.
The statistical analysis of the mobility paths suggests the existence of a travel time budget for the mobility and points out the role of the rest times in the empirical relation between the mobility time and the corresponding path length.
We succeed to highlight two connected mobility subnetworks extracted from the whole road network, that are able to explain the majority of the observed mobility.
Our approach shows the existence of characteristic mobility paths in Venice for the tourists and for the residents.
Moreover the data analysis highlights the different mobility features of the considered case studies and it allows to detect the mobility paths associated to different points of interest.
Finally we have disaggregated the Italian and foreigner categories to study their different mobility behaviors.

\section*{The datasets}

The dataset used in this study has been provided by the Italian mobile phone company TIM and contains geo-referenced positions of tens of thousands anonymous devices (e.g. mobile phones, tablets, etc. ...), whenever they performed an activity (e.g. a phone call or an Internet access) during eight days from 23/2/2017 up to 02/03/2017 (Carnival of Venice dataset), and from 14/7/2017 up to 16/7/2017 (\emph{Festa del Redentore} dataset).
According to statistical data, 66\% of the whole Italian population has a smart-phone and TIM is one the greatest mobile phone company in Italy whose users are $\sim30\%$ of the whole smart-phone population.
The datasets refer to a geographical region that includes an area of the Venice province, so that it is possible to distinguish commuters from sedentary people and the different transportation means used to reach Venice.
Each valid record gives information about the GPS localization of the device, the recording time, the signal quality and also the roaming status, which in turns allow to distinguish between Italian and
foreigners.
The devices are fully anonymized and not reversible identification numbers (ID) are automatically provided by the system for mobile phones and calls within the scope of the trial; the ID is kept for a period of 24 hours.
During each activity a sequence of GPS data is recorded with a 2 sec. sampling rate and the collection stops when the activity ends.
As matter of fact during an activity most of people reduce their mobility except if they are on a transportation mean, so that the dataset contains a lot of small trajectories that have to be joined to reconstruct the daily mobility.
After a filtering procedure these data provide information on the mobility of a sample containing 3000~–~4000 devices per day.
Since the presences during the considered events were of the order of 105 individuals per day, as reported by the local newspapers, we estimate an overall penetration of our sample of 3~–~4\%.
The filtering procedure and the other statistical informations about the sample penetration are discussed in the original paper~\cite{Mizzi2018}.

\section*{Mobility paths reconstruction on the road network}

The procedure of mobility path reconstruction considers separately the land mobility and the water mobility since the two mobility networks have different features, so that it is necessary to check carefully the transitions from one network to the other.
To create a mobility path, we connect two successive points left by the same device using a best path algorithm on the road network with a check on the estimated travel speed to avoid unphysical situations and discarding the paths whose velocity is clearly not consistent with
the typical pedestrian velocity (or ferryboat velocity).
To end a land path and to start a water path, we require that at least two successive points of the same device are attributed to a ferryboat line by the localization algorithm.
In the case of a single point on a ferryboat line, we force the localization of this point on the nearest road on the land.

The reconstruction of the mobility paths also allows to study how people perform their mobility on the road network.
We consider the problem of determining the most used subnetwork of the Venice road network.
The existence of mobility subnetworks could be the consequence of the peculiarity of Venice road network, where it is quite easy to get lost
if you do not have a map.
Therefore people with a limited knowledge of the road network move according to paths suggested by Internet sites or following the signs on the roads.
To point out a mobility subnetwork we rank the roads of Venice according to a weight proportional to the number of mobility paths passing through each road.
Thus We define a relevant subnetwork as a connected subnetwork that explains a considerable fraction of the observed mobility.
In this case each road (identified by two nodes in the poly-line format) represents the link of our weighted graph and we can apply the DNetPRO technique shown in~\ref{implementation:network} to identify the network core with only closed paths\footnote{
  Pendant nodes are unphysical solutions in our model since we are interested on the pedestrian mobility paths that bring people from one location to an other.
}.

% \begin{center}
% \begin{figure}[htbp]
% \centering
% \includegraphics[width=0.85\textwidth]{venice_step.pdf}
% \caption{From top-left to right-bottom, we plot four mobility subnetworks with increasing number of roads, selected by the DNetPRO algorithm using the Carnival dataset.
% }
% \label{fig:venice_step}
% \end{figure}
% \end{center}

Starting from the previously evaluated daily flows for each road, we order in a decreasing way the roads according to the observed
flows.
The DNetPRO algorithm scrolls down the list adding the road to a temporary list.
At every step the \quotes{pruning process} starts on the selected roads cutting the isolated roads in order to get a connected subnetwork\footnote{
  Since we are interested on the largest connected component the \emph{merging} parameter is off.
}.
Therefore the number of nodes of the subnetwork increases in a discontinuous way, when the adding of a new road in the list allows to
connect several previously selected roads.
After several parametric scans, we found that the best result for our purposes is achieved by choosing about the 10\% of the nodes in the whole Venice road network.
In the Fig~\ref{fig:venice_step} we show four consecutive selected subnetworks in the case of Carnival dataset to illustrate how the algorithm operates.

% \begin{center}
% \begin{figure}[htbp]
% \centering
% \includegraphics[width=0.85\textwidth]{venice_result.pdf}
% \caption{Picture (a): selected subnetworks (highlighted in purple) from the road network of the Venice historical centre (in the background), that explain 64\% of the recorded mobility in the datasets.
% The top picture refers to the Carnival mobility during 26/02/2017 and corresponds to 13\% of the total length of the Venice road network.
% The picture (b) refers to the \emph{Festa del Redentore} mobility during 15/07/2017 and corresponds to 15\% of the total length of the Venice road network.
% }
% \label{fig:venice_result}
% \end{figure}
% \end{center}

Using the DNetPRO algorithm we are able to extract a subnetwork which explains the 64\% of the observed mobility using 13\% of the total road network length for the case of the Carnival dataset and 15\% of the total length in the case of the \emph{Festa del Redentore} dataset.

The selected road subnetworks are plotted in Fig~\ref{fig:venice_result} for both the datasets.
As a matter of fact, many of the highlighted paths are also suggested by Internet sites.
However, we remark some differences that can be related by the different nature of the considered events.
During the Carnival of Venice the mobility seems to highlight three main directions connecting the railway station and the \emph{Piazzale Roma} (top-left in the map), which are the main access points to the Venice historical centre, with the area around San Marco square, where many activities where planned during 26/02/2017.
In the case of the \emph{Festa del Redentore} the structure is more complex due to the appearance of several paths connecting the station and \emph{Piazzale Roma} with the \emph{Dorsoduro} district in front of the \emph{Giudecca} island.

This geometrical structure could have a double explanation: on one hand the \emph{Festa del Redentore} introduces an attractive area near the \emph{Giudecca} island, where the fireworks take place in the evening; on the other hand the \emph{Festa del Redentore} is a festivity very much felt by the local population, that knows the Venice road network and performs alternative paths.

On these subnetwork we also map the mobility of Italians and foreigns separately.
The results of this application are deeply discussed in the paper.

\end{document}
