\documentclass{standalone}

\begin{document}

\chapter*{Appendix B - Venice Road Network}\addcontentsline{toc}{chapter}{Appendix B - Venice Road Network}
\markboth{Appendix B}{Venice Road Network}

Tourist flows in historical cities are continuously growing in a globalized world and adequate governance processes, politics and tools are necessary in order to reduce impacts on the urban livability and to guarantee the preservation of cultural heritage.
The ICTs offer the possibility of collecting large amount of data that can point out and quantify some statistical and dynamic properties of human mobility emerging from the individual behavior and referring to a whole road network.
In this work we analyze a new dataset that has been collected by the Italian mobile phone company TIM, which contains the GPS positions of a relevant sample of mobile devices when they actively connected to the cell phone network.
Our aim is to propose innovative tools allowing to study properties of pedestrian mobility on the whole road network.
Venice is a paradigmatic example for the impact of tourist flows on the resident life quality and on the preservation of cultural heritage.
The GPS data provide anonymized geo-referenced information on the displacements of the devices.
After a filtering procedure, we develop specific algorithms able to reconstruct the daily mobility paths on the whole Venice road network.
The statistical analysis of the mobility paths suggests the existence of a travel time budget for the mobility and points out the role of the rest times in the empirical relation between the mobility time and the corresponding path length.
We succeed to highlight two connected mobility subnetworks extracted from the whole road network, that are able to explain the majority of the observed mobility.
Our approach shows the existence of characteristic mobility paths in Venice for the tourists and for the residents.
Moreover the data analysis highlights the different mobility features of the considered case studies and it allows to detect the mobility paths associated to different points of interest.
Finally we have disaggregated the Italian and foreigner categories to study their different mobility behaviors.

\end{document}
