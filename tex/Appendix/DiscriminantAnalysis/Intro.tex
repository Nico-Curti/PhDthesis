\documentclass{standalone}

\begin{document}

\chapter*{Appendix A - Discriminant Analysis}\addcontentsline{toc}{chapter}{Appendix A - Discriminant Analysis}
\markboth{Appendix A}{Discriminant Analysis}

The classification problems aim to associate a set of \emph{pattern} to one or more \emph{classes}.
With \emph{pattern} we identify a multidimensional array of data labeled by a pre-determined tag.
In this case we talk about \emph{supervised learning}, i.e the full set of data is already annotated and we have prior knowledge about data association to the belonging classes.
Since in this work only supervised learning algorithms have been analyzed we do not cite other different learning methods.

In machine learning a key rule is assumed by Bayesian methods, i.e methods which use a Bayesian statistical approach to the analysis of data distributions.
It can be proof that if the distributions under analysis are known, i.e a sufficient number of moments of it is known with a sufficient precision, the Bayesian approach is the best possible method to face on the classification problem. % cite!

\end{document}
