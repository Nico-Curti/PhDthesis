\documentclass{standalone}

\begin{document}

\chapter*{Appendix A - Discriminant Analysis}\addcontentsline{toc}{chapter}{Appendix A - Discriminant Analysis}
\markboth{Appendix A}{Discriminant Analysis}

The classification problems aim to associate a set of \emph{pattern} to one or more \emph{classes}.
With \emph{pattern} we identify a multidimensional array of data labeled by a pre-determined tag.
In this case we talk about \emph{supervised learning}, i.e the full set of data is already annotated and we have prior knowledge about the association between data and classes.

In machine learning a key rule is played by Bayesian methods, i.e methods which use a Bayesian statistical approach to the analysis of data distributions.
It can be proved that, if the underlying distributions are known, i.e a sufficient number of its moments are known with a sufficient precision, the Bayesian approach is the best possible method to face the classification problem (\emph{Bayesian error rate}\cite{Fukunaga:1990:ISP:92131}).

\end{document}
