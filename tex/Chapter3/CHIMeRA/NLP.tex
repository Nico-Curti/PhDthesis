\documentclass{standalone}

\begin{document}

\section[NLP]{Natural Language Processing}\label{chimera:nlp}


\begin{center}
\begin{figure}[htbp]
\centering
\includegraphics[width=\textwidth]{pubmed_nlp.png}
\caption{Number of publications containing the sentence \quotes{natural language processing} in PubMed in the period 1978–2018.
As of 2018, PubMed comprised more than 29 million citations for biomedical literature.
}
\label{fig:pubmed_nlp}
\end{figure}
\end{center}


Natural Language Processing (NLP) is a quite novel research field driven by the increasing availability of textual data (ref. Fig.~\ref{fig:pubmed_nlp}).
As told in the previous sections the incoming of Internet world exponentially increase the amount of data shared by people and the major part of them are textual data, i.e data composed by words, phrases and more in general texts.
The NLP joins together techniques coming from the linguistic, computer science, information theory and artificial intelligence researches and it concerns the interactions between human languages and computers, or in other words it studies how a computer can analyze a huge amount of natural language data and how it could extract numerical informations from them.
This is a very hard task to perform since it is not straightforward to teach to a machine how humans communicate between them so a key role is played by the artificial intelligence researches in the developing of new algorithmic techniques.
The final purpose of the NLP is, in fact, to read, decipher, understand and make sense of the human languages extracting valuable and numerical results.

Most of the modern NLP techniques are based on a Machine Learning approach to the problem and thus we can find statistical methods against deep learning neural networks trained to face on these kind of problems.
A first step has to be performed to convert the human speech into a machine readable input; then the audio signal is converted into a string text and only at the end the text can be analyzed from the machine.
Applying this work-flow in forward and reverse mode we can perform a communication between a human and a machine and vice versa.
In this section we will ignore how the conversion from human voice to numerical inputs could be performed and its related problems and solutions but we will focusing on the last part of this pipeline, i.e in the description of the most common techniques to process a string text into numerical values.
This is also the case related to our CHIMeRA project, in which we have a huge amount of names and strings related to medical terms and we want to standardize and increase their overlap.

First of all we have to take care that each human language has its own characteristics and thus it is harder to create to a pipeline ables to process all the languages at the same time while it is easier to tune an algorithm on a particular language.
In our work we were focused on the Italian language (SymptomsNet) and on the English language (CHIMeRA Network).
Since the SymptomsNet project was developed as simple proof of concepts, the developed Italian pipeline was really naive and for sake of brevity we will focus only on the CHIMeRA pipeline, i.e the English one.
We would stress that in our application we were not interested on the understanding of the words meaning but we want to minimize the word heterogeneity maximizing their overlap.
In this way we can ignore the semantic meaning of the strings and we could focus only on their syntaxes.

Syntax refers to the arrangement of words to make them grammatical sense.
In this way we can create group of words applying grammatical rules: the grammatical rules have to be converted into algorithms which take in input a word and they give in output a processed version of the same word.
In this case there is not a numerical output but just a reorganization of the string letters and words.
The most common techniques involved in the syntax analysis are:

\begin{itemize}

  \item \textbf{Lemmatization:} it reduces the inflectional forms of a word into a single form.
  \item \textbf{Morphological segmentation:} it splits words into individual units called morphemes.
  \item \textbf{Word segmentation (tokenization):} it splits a large set of continuous text into units.
  \item \textbf{Part-of-speech tagging:} it identifies the grammatical part of speech for every word.
  \item \textbf{Parsing:} it provides the grammar analysis of the provided sentence.
  \item \textbf{Sentence breaking:} it divides a continuous text into sentences placing boundaries.
  \item \textbf{Stemming:} it cuts the inflected words to their root form.

\end{itemize}

Combinations of these algorithms can be find in everyday applications starting from email assistants or website chat box to the more advanced sentimental analyses and fake news identifiers.
NLP pipelines could be used also in biomedical applications and the modern multinational factories like Amazon, IBM or Google are financing different kinds of research on this topic.
\href{https://aws.amazon.com/it/comprehend/medical/}{Amazon Comprehend Medical} is a NLP service developed by Amazon to extract disease conditions, medications and treatment outcomes from patient notes, electronic health records and other clinical trial reports.
At the same time also companies like Yahoo and Google based their filters and email classifiers on NLP algorithms to stop spam.
Also the fake news hot topic of the these years is faced on by NLP pipeline and the NLP Group at MIT is developing new tools to determine if a source is accurate or politically biased based on analyses of texts.

In our applications we constructed a custom pipeline based on part of these algorithms.
In the following sections we will describe in detail our pipeline which was tuned for our case study: we would stress that the efficiency of our pipeline could not be generalized to other datasets since our purpose was to obtain the best result for our applications.
In other words we can say that we had over-fitted our data.
Moreover we have to clarify that our pipeline is not fully-automatic but it was made according to a semi-supervised approach: we customize the work-flow following the issues showed by our applications.


\end{document}
