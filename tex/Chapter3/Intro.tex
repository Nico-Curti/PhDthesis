\documentclass{standalone}

\begin{document}

\chapter[Big Data]{Biological Big Data - CHIMeRA project}\label{chapter3:bigdata}

The increasing availability of large-scale biomedical literature under the form of public on-line databases, has opened the door to a whole new understanding of multi-level associations between genomics, protein interactions and metabolic pathways for human diseases via network approaches.
Many structures and resources aiming at such type of analyses have been built, with the purpose of disentangling the complex relationships between various aspects of the human system relating to diseases~\cite{SymtomsNet, HumanPhenotype}.


%talk about data structured and unstructured
%Many public datasets available.
%Description of the database used in chimera.
%Problems about the intersections and partial informations (single db).


\end{document}
