\documentclass{standalone}

\begin{document}

\section[Image Quality]{Image Quality}\label{quality}

The most common image quality evaluator is given by our eyes.
This is true also for SR problems: the final purpose still remain to obtain images that are better visible for human eyes, the so called \emph{visual loss}.
We can however provide some mathematical formulas which allows to quantitative evaluate the image quality.
In both cases we need to establish a relation between the original image and the produced one.
Thus we can formulate a quality score only with a reference image.
In SR problems, or more in general in up-sampling problems, we can compare the original HR image with the image obtained by the output of our model.
In this way our quality score will be a measure of similarity between the two images.

The simple similarity score can be obtained evaluating the peak-signal-to-noise-ratio (PSNR).
This quantity is commonly used to establish the compression lossless of an image and it can be computed as

$$
PSNR = 20 \cdot \log_{10}\left( \frac{\max(I)}{\sqrt(MSE)} \right)
$$
\\
where $\max(I)$ is the maximum value which can be taken by a pixel in the image (in general it will be 1 or 255 depending on the image format chosen) and $MSE$ is the Mean Square Error (ref. \ref{cost}) between the original image and the reconstructed one.
The MSE for an image can be computed as:

$$
MSE = \frac{1}{WH} \sum_{i=1}^{W}\sum_{j=1}^{H} \left( I(i, j) - K(i, j) \right)^2
$$
\\
where $W$, $H$ are width and height of the two images and $I$, $K$ are the original and reconstructed images, respectively.

In other words the PSNR is the maximum power of the signal over the background noise.
It is expressed in decibel (dB) because the image values ranging in a wide interval and the logarithmic function rearrange the domain.
The PSNR is probably the most common quality score~\ref{psnr-ssim} but it does not always related to a qualitative visual quality.
Despite it is commonly used as loss function for SR models.

Considering the series of images shown in Fig.\ref{figure:resampling} we can evaluate the PSNR score considering as high-quality image the bicubic and Lanczos results.
In this way we can compare the nearest interpolation result against them.
In the down-sampling case the PSNR of the nearest algorithm against the bicubic one is equal to 21.9 while against the Lanczos result we obtain a value of 21.8.
The PSNR value obtained with the Lanczos algorithm is less than the result obtained with the bicubic interpolation as expected from a mathematical point-of-view since the nearest interpolation should be more similar to the bicubic one.
We can reach analogous conclusions in the up-sampling case...


% Down Bicubic PSNR : 21.903
% Down Bicubic SSIM : 0.775
% Down Lanczos PSNR : 21.849
% Down Lanczos SSIM : 0.771
% Up   Bicubic PSNR : 31.162
% Up   Bicubic SSIM : 0.931
% Up   Lanczos PSNR : 31.299
% Up   Lanczos SSIM : 0.930
% Up   Lanczos PSNR : 45.199
% Up   Lanczos SSIM : 0.995
% DOWN Lanczos PSNR : 44.167
% DOWN Lanczos SSIM : 0.999


\end{document}
