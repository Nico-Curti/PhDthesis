\documentclass{standalone}

\begin{document}


\section[Compare dataset]{SNP classification}\label{rfbp:snp}

The few available applications of the rFBP algorithm to real data are amenable to two aspects: I) learning technique; II) algorithm implementation.
The first one is related to the intrinsic definition of the algorithm which is designed to reach a complete memorization of the training dataset; in the other Machine Learning processes we normally want to avoid this kind of results since it could bring to \emph{over-fitting} problems.
The second one is given by the binary values involved in each step of the algorithm which intrinsically limit the possible applications\footnote{
  The Neural Network weights can assume only binary values since they model up/down spins.
  Moreover also the input is required to be a spin configuration and thus binary.
  The common Machine Learning problems involve floating-point values as input pattern.
}.


Classification problems which involved only binary quantities are quite small but the GWA is one of them.
In the GWA we have a series of genome data belonging to different classes as input.
A genome is the ensemble of genes of an organism and each gene is identified by a series of nucleotides with 4 possible values (G, guanine; C, cytosine; A, adenine; T, thymine).
The comparison between a reference (healthy) genome and an infected one highlights the biological mutation related to the understudy disease.
This mutation are the so-called SNPs (Single Nucleotide Polymorphisms).
So we can identify a genome as a sequence of its mutation in relation to a reference genome, i.e a sequence of two possible values given by the on/off of the mutation in each nucleotide.

The COMPARE project aims to develop new methods to avoid the genetic disease transmission.
In this project plays a crucial role the \emph{Source Attribution}, i.e the classification of a given disease based on the list of its mutation.

We tested the rfBP on $210$ Salmonella enterica genome sequences, $4857450\,bp$ (base pairs) long, living inside animals.
Our early goal was to discriminate those bacteria living in pigs (159 samples) with respect to all the others animals (51 samples).

First of all we filter our data removing from each genome a base if it is not mutated in each sample.
In this way we reduce the number of bases to $8189\,bp$.
A graphical representation of these samples is given in Fig.~\ref{fig:SNPsAle}.
The dataset was divided in training and test sets using a stratified cross-validation procedure to guarantee a proportional subdivision of the samples into the two classes.
The algorithm hyper-parameters was tuned on the training set based on the performances obtained using a internal stratified 10-fold cross-validation: in each fold the training was performed by a given sequence of hyper-parameters and the performances evaluated on the corresponding test set; the hyper-parameters configuration which obtains the best performances on the full training set was chosen as best configuration.
The performances evaluation was performed using the custom \emph{Scorer} library.
Considering the unbalanced sample quantities the Matthews Correlation Coefficient (MCC) is chosen as good scorer indicator for the evaluation.

With the tuned hyper-parameters we performed the training of rFBP algorithm on different percentage of the training set: $25\%$, $45\%$, $65\%$ and $85\%$.
In the same way we train also a list of the most common Machine Learning classifiers: single perceptron with floating-point weights (Perc); standard Neural Network with gradient descent as updating rule (MLP); support vector machine with linear kernel (lSVM); support vector machine with radial kernel (rSVM); linear discriminant analysis (LDA); decision tree (DT); random forest (RF); k-nearest neighbors with 2-clusters (kNN); Guassian process (GP); diag-quadratic discriminant analysis (GNB); Bernoulli naive bayes (BNB); AdaBoost (AdaB).
For each training percentage we perform the optimization of the hyper-parameters of each classifier with the same number of optimization steps.
In Fig.~\ref{fig:confronto_bestclassificatoriACC, fig:confronto_best_classificatoriMCC} the accuracies and MCC results are shown, respectively.

From this analysis we can conclude that the rFBP algorithm shows comparable performances with the other classifiers.
These performances globally grow with the training set size but only the rFBP is able to reach a \quotes{perfect learning} configuration, i.e accuracy of 100\% and MCC=1.
We have also noticed that the rFBP classifier and the GNB are the only two algorithms which qualitatively does not show performances saturation on their training.

A second analysis was performed on the data distribution using a multiple $\chi^2$-test.
Starting from the whole set of genomes we can compute the contingency-matrix of the two classes\footnote{
  The contingency-matrix displays the (multivariate) frequency distribution of the variables.
  Each row will count the number of hosts with/without the SNPs.
  Each column will identify a class.
}.
The $\chi^2$-test was performed on the full set of $8189\,bp$ and so the extracted \emph{p-values} were corrected according multiple-tests.
Using the \v{S}id\'ak~\cite{Sidak1967} correction method and by the definition of significant threshold of $0.05$ we found $1103$ significant bases.
An analogous $\chi^2$-test was performed on the rFBP weights to identify a putative

% MISS

\end{document}
