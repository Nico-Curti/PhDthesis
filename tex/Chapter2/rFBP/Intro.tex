\documentclass{standalone}

\begin{document}

\section[rFBP]{Replicated Focusing Belief Propagation}\label{rfbp}

Until now we are talking about neural networks based on the standard update rule of backpropagation.
Other learning rule for weight updates were proposed and the choice of the best one it is a still open-problem.
The final purpose is to obtain a feasible learning rule able to model the biological learning of the human brain.

The learning problem could be faced on through statistical mechanic models joined with the so called Large Deviation Theory.
In general the learning problem can be split in two sub-parts: the classification problem and the generalization one.
The first aims to completely store a pattern sample, i.e a prior known ensemble of input-output associations (\emph{perfect learning}).
The second one corresponds to compute a discriminant function based on a set of features of the input which guarantees a unique association of a pattern.

From a statistical point-of-view many Neural Network models have been proposed and the most promising seem to be models based on spin-glasses.
Starting from a balanced distribution of the system, generally based on Boltzmann distribution, and under proper conditions, we can proof that the classification problem became a NP-complete computational problem.
A wide range of heuristic solution to that type of problem were proposed.

In this section we show one these algorithms developed by Zecchina et al.~\cite{BaldassiE7655} and called \emph{Replicated Focusing Belief Propagation} (rFBP).
The theoretical background of the algorithm is beyond the scope of this thesis so we focus on its numerical implementation and optimization.

Moreover, despite their proofed theoretical efficiency, the applications on real data are still few.
Thus we show the application of the optimized version of the rFBP algorithm on the Genome Wide Association (GWA) data provided by the European \href{https://www.compare-europe.eu/}{COMPARE project}.
This work was also presented on the CCS-Italy (Conference of Complex System) of the 2019~\cite{DallOlioCCS19}.

\end{document}
