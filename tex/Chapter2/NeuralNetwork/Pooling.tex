\documentclass{standalone}

\begin{document}

\section[Pooling function]{Pooling function}\label{pooling}

Output Neural Network feature maps often suffer of sensitivity on feature location in the input.
One possible approach to overcome this problem is to down sample the feature maps making the resulting feature map more robust to changes in the position.
Pooling functions perform this kind of down sample and they reduce the spatial dimension (but not depth) of the input.
Their use represents an important computational performance improver tool (less feature, less operations) and a useful dimensionality reduction method.
The reduction of feature quantity can also prevent over-fitting problems and it improves the classification performances.

Pooling layers are intrinsically related to Convolutional layers.
The analogy lives in the filter mapping procedure which produces the output in both methods.
While in the Convolutional layer we map a filter over the input signal and we apply a multiplication of the layer weights and the signal values, in the pooling layer we simply change the filter function keeping the same filter mapping procedure (see section~\ref{convolutional} for more informations).
The input parameters of the method are the same of the Convolutional one:

\begin{itemize}

\item Input: (batch, width, height, channels) of the input data.

\item Stride: scalar which control the amount of pixels that the filter slide.

\item K: (kx, ky) window filter size.

\end{itemize}

\end{document}
