\documentclass{standalone}

\begin{document}


\section[Convolutional]{Convolutional function}\label{convolutional}

Convolutional Neural Network (CNN) are particularly designed for image analysis.
Convolution is the mathematical integration of two functions in which the second one is translated by a given value.

In signal processing this operation is also called \emph{crossing correlation} ad it is equivalent to the \emph{autocorrelation} function computed in a given point.
In image processing the first function is represented by the image $I$ and the second one is a kernel $k$ (or filter) which shift along the image.
In this case we will have a 2D discrete version of the formula given by:

$$
C = k * I
$$

$$
C[i, j] = \sum_{u=-N}^{N} \sum_{v=-M}^{M} k[u, v] \cdot I[i - u, j - v]
$$

where $C[i, j]$ is the pixel value of the resulting image and $N, M$ are kernel dimensions.

The use of CNN in modern image analysis applications can be traced back to multiple causes.
First of all the image dimensions are increasingly bigger and thus the number of variables/features, i.e pixels, is often too big to manage with standard DNN\footnote{
  If we consider a simple image $224\times224$ with $3$ color channels we obtain a set of $150'528$ features.
  A classical DNN layer with this input size should have $1024$ nodes for a total of more than $150$ million weights to tune.
}.
Moreover if we consider detection problems, i.e the problem of detecting an set of features (or an object) inside a larger pattern, we want a system able to recognize the object regardless of where it appears into the input.
In other words, we want that our model would be independent by simple translations.

Both the above problems can overcome by CNN models using a small kernel, i.e weight mask, which maps the full input.
A CNN is able to successfully capture the spatial and temporal dependencies in an signal through the application of relevant filters.


%https://victorzhou.com/blog/intro-to-cnns-part-1/


\end{document}
