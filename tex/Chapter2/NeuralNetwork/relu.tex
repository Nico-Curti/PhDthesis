\documentclass{standalone}

\begin{document}


\subsection[Relu]{Rectified Linear Unit}\label{NN:relu}

The ReLU (Rectified Linear Unit) activation functions are the most used into the modern Neural Networks models.
Their diffusion is imputed to their numerical efficiency and to the benefits they bring~\cite{Glorot2011Relu}:

\begin{itemize}

\item Information disentangling: the main purpose of a Neural Network model is to tune a discriminant function able to associate a set of input to a prior-known output classes.
A dense information representation is considered \emph{entangled} because small differences in input highly modifies the data representation inside the network.
On the other hand, a sparse representation tends to guarantee a conservation of the learning features.

\item Information representation: different inputs can lead different quantities of useful informations.
The possibility to have null values in output (ref Tab.~\ref{tab:activations}) allows a better representation of the representation dimension inside the network.

\item Sparsity: sparsity representation of data are exponentially efficient in comparison to dense ones, where the exponential power is given by the number of no-null features~\cite{Glorot2011Relu}.

\item Vanish gradient reduction: if the activation output is positive we have a no-bound gradient value.
%add other considerations

\end{itemize}

\end{document}
