\documentclass{standalone}

\begin{document}

\section[Object Detection]{Object Detection}\label{obj_detection:obj}

Object detection is one of the larger deep learning sub-discipline, especially when we talk about Neural Network models.
This kind of problems aim to identify single or multiple objects into a picture or video stream.
The possible applications of these tools are everywhere these days and they involve object tracking, video surveillance, pedestrian detection, anomaly detection, people counting, self-driving cars or face detection and the list goes on.

There are many machine learning and deep learning techniques and algorithms proposed during the years about this topic and each one has its pros and cons.
The most prominent and moder techniques involve the use of very deep Neural Network models with a huge amount of parameters to tune.
The most famous one are probably the Faster R-CNN (\emph{Faster Region Convolutional Neural Network})~\cite{ren2015faster} and their \quotes{evolution} into the YOLO (\emph{You Only Look Once}) model~\cite{redmon2015look, redmon2016yolo9000, redmon2018yolov3}.

The R-CNN models are one of the state-of-art CNN-based deep learning object detection model and their evolution into Fast R-CNN tries to improve their speed.
The standard approach for object detection is based on moving a \emph{sliding window} to search in every position of the image the looking for objects.
However, the intrinsic problem of these kind of method are the window dimensions and the large computation required to map with multiple window sizes the full image.
Different objects, or even the same kind of objects, could have different aspect ratios and sizes in relation to the position of the camera which captured the image or to their distances.
R-CNN models try to overcome these problems generating about 2k region proposals, i.e bounding boxes, and applying to each one a image classification using standard CNN.
Finally, each detected region can be refined using a regression approach.

A Faster R-CNN model is based on the same idea but, instead of feeding the bounding boxes to the CNN, it feeds the input image to the CNN to generate a convolutional feature map.
Starting from this feature map we can easier identify the region of proposals (Region Proposal Network) and warp them into squares.
The list of these regions are then reshaped using a Polling layer and processed by a fully connected layer.
The advantages of Faster R-CNN are thus visible: we do not need to feed 2k region proposals to the CNN every time but the feature map is generate once per image using the convolution operation.
In this way we can also separate the feature map creation to the selective search algorithm.

A key role on these models is given by the \emph{anchor} concept: an \emph{anchor} is essentially a box and it identifies the shape of a portion of the input image at different scale levels.
The CNN feature map feeds the Region Proposals Network which uses a sliding window over it, generating $k$ anchor boxes.
These boxes are certainly fewer than the 2k previous cited windows.

A breakthrough idea on the real-time object detection was the introduction of the YOLO model.
The model was developed by Redmon et al. at Washington University and it is probably the state-of-art on object detection, especially for its very incredible speed (it can reach 45 FPS on modern GPUs!).
Certainly it is the faster method publicly available, but its popularity is due also to its innovative strategy in object detection.
Despite all the other algorithms use regions to localize the object into the image, the YOLO network does not look at the complete image but only on a parts of it which has the higher probability to contain an object.
In YOLO a single CNN predicts the bounding boxes and the class probabilities of them.
YOLO slits a single image into a $S\times S$ grid and on each grid $m$ bounding boxes are taken.
For each of them, the CNN outputs a class probability and offset values.
Finally, these bounding boxes are filtered according to their probability and a chosen threshold.

One of the most bigger limitation of this model is that it struggles with small objects.
This is due to the spatial constraints of the algorithm.
Fortunately, in the previous section we have already discussed on how we can overcome this kind of problem using Super Resolution.
In the next section we will discuss about further characteristics of the YOLO model and about its implementation into the \textsf{Byron} library, considering its efficiency against the original implementation.
Finally, we will join the efficiency of the previous Super Resolution models to the performances of our optimized implementation of YOLO.


%Introduction on the image classification and detection with Yolo architecture.
%Implementation in Byron with description of performances against darknet (original implementation).
%Focus on performances (time, memory, cpu).


\end{document}
