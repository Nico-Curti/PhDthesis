\documentclass{standalone}

\begin{document}

\chapter*{Introduction}\label{Introduction}\addcontentsline{toc}{chapter}{Introduction}
\markboth{Introduction}{Introduction}

Biomedical data are growing both in size and breath of possible uses.
Of special importance are the so called biomedical big data, blanket term describing data generated from several machines and used to describe the health state of a person:
\begin{enumerate}
\item\textbf{Next generation sequencing NGS} NGS technology. RNA-seq: experimental procedure, challenges and opportunities in statistical data analysis. ChIP-Seq: experimental procedure and statistical data analysis.
\item\textbf{Proteomics and Metabolomics} LC/MS technology, challenges in data processing. Biological pathways.
\item\textbf{Biomedical imaging} Imaging techniques, acquisition methods and data structures/characteristics for different imaging modalities.
\item\textbf{Statistical Analysis of Imaging Data} Data processing techniques, study designs, analysis strategies, research questions and goals. Radiomics.
\item\textbf{Brain Networks and Imaging Genetics} The importance of brain networks in differentiating between healthy and mentally ill subjects, methods on how to estimate the brain network which may or may not rely on additional clinical, demographic and genetic information.
\item\textbf{Molecular genetics and population genetics} Biological backgrounds for statistical genetics, concepts from population genetics that are most relevant to association analysis.
\item\textbf{Genetic association studies} Tests for association, challenges especially in the context of genome-wide association studies (GWAS), including how to correct for population stratification and multiple testing.
\end{enumerate}

These datasets are known to contain vast amount of information, especially when connected together to enhance the power of the biological modeling~\cite{Pooley2005, Castellani2016}.


% take something from the EuroPar18 introduction

% in questo lavoro si affronteranno diverse tematiche relative alla Big Data Analytics e si propongono soluzioni inerenti ad ognuna di esse con esempi sviluppati ed applicati a dati reali.
% Partendo dalla curse of dimensionality e la feature extraction (dnet), passando per la visualizzazione dei dati con le NN fino alla eterogeneità dei dati (chimera)

% definire feature come variable e dire che nel resto del testo verranno usati in maniera indistinta i due termini

\end{document}
