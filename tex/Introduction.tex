\documentclass{standalone}

\begin{document}

\chapter*{Introduction}\label{Introduction}

in questo lavoro si affronteranno diverse tematiche relative alla Big Data Analytics e si propongono soluzioni inerenti ad ognuna di esse con esempi sviluppati ed applicati a dati reali.
Partendo dalla curse of dimensionality e la feature extraction (dnet), passando per la visualizzazione dei dati con le NN fino alla eterogeneità dei dati (chimera)

definire feature come variable e dire che nel resto del testo verranno usati in maniera indistinta i due termini

\end{document}
